\documentclass[11pt]{article}
\usepackage{amssymb,amsmath,lmodern,slashed,cancel}
\usepackage{array}
\usepackage{bm}
\usepackage{graphicx}
\usepackage[hidelinks]{hyperref}
\usepackage[toc,page]{appendix}
\usepackage{footnote,url}
\usepackage{calc}
\usepackage{caption}
\usepackage[export]{adjustbox}
\usepackage{subcaption}
\usepackage[margin=.5in,top=1in]{geometry}
\usepackage{fancyhdr}
\pagestyle{fancy}
\lhead{Sid Narayanan}
\rhead{NUPAX Part III}
\chead{\today}
\renewcommand{\headrulewidth}{0.4pt}
\renewcommand{\footrulewidth}{0.4pt}
\newlength\dlf
\newcommand\alignboxed[2]{
  &
  \begingroup
  \settowidth\dlf{$\displaystyle #1$}
  \addtolength\dlf{\fboxsep+\fboxrule}
  \hspace{-\dlf}
  \boxed{#1 #2}
  \endgroup
}
\newcommand\numberthis{\addtocounter{equation}{1}\tag{\theequation}}
\newcommand{\ve}{\mathbf{v}}
\newcommand{\Se}{\mathbf{S}}
\newcommand{\kk}{\vec{k}}
\newcommand{\ue}{\bm{u}}
\newcommand{\sgn}{\text{sgn}}
\newcommand{\p}{\vec{p}}
\newcommand{\pop}{\hat{p}}
\newcommand{\x}{\vec{x}}
\newcommand{\xop}{\hat{x}}
\newcommand{\w}{\mathbf{w}}
\newcommand{\z}{\mathbf{z}}
\newcommand{\y}{\vec{y}}
\newcommand{\qq}{\vec{q}}
\newcommand{\tv}{\tilde{\ve}}
\newcommand{\vL}{\vec{L}}
\newcommand{\vS}{\vec{S}}
\newcommand{\vso}{\vec{S}^{(1)}}
\newcommand{\vst}{\vec{S}^{(2)}}
\newcommand{\so}{S^{(1)}}
\newcommand{\st}{S^{(2)}}
\newcommand{\vJ}{\vec{J}}
\newcommand{\vl}{\vL}
\newcommand{\vj}{\vJ}
\newcommand{\vjo}{\vec{J}^{(1)}}
\newcommand{\vjt}{\vec{J}^{(2)}}
\newcommand{\jo}{J^{(1)}}
\newcommand{\jt}{J^{(2)}}
\newcommand{\vs}{\vS}
\newcommand{\proj}{\text{\proj}}
\newcommand{\psibar}{\bar{\psi}}
\newcommand{\Psibar}{\bar{\Psi}}
\newcommand{\ubar}{\bar{u}}
\newcommand{\nubar}{{\bar{\nu}}}
\newcommand{\vbar}{\bar{v}}
\newcommand{\qbar}{\bar{q}}
\newcommand{\fbar}{\bar{f}}
\newcommand{\dbar}{\bar{d}}
\newcommand{\bbar}{\bar{b}}
\newcommand{\cbar}{\bar{c}}
\newcommand{\ebar}{\bar{e}}
\newcommand{\tr}{\text{Tr}}
\newcommand{\spane}{\text{span}}
\newcommand{\ra}{\rangle}
\newcommand{\la}{\langle}
\newcommand{\plus}{|+\ra}
\newcommand{\sulp}{\la+|}
\newcommand{\minu}{|-\ra}
\newcommand{\unim}{\la-|}
\newcommand{\pp}[2]{\dfrac{\partial #1}{\partial #2}}
\newcommand{\ppt}[2]{\dfrac{\partial^2 #1}{\partial #2^2}}
\newcommand{\dd}[2]{\dfrac{d #1}{d #2}}
\newcommand{\fdd}[2]{\dfrac{\delta #1}{\delta #2}}
\newcommand{\fddt}[2]{\dfrac{\delta^2 #1}{(\delta #2)^2}}
\newcommand{\ddt}[2]{\dfrac{d^2 #1}{d #2^2}}
\newcommand{\dre}{\dot{\re}}
\newcommand{\vmu}{\vec{\mu}}
\newcommand{\inv}{^{-1}}
\newcommand{\Mme}{\mathcal{M}}
\newcommand{\La}{\mathcal{L}}
\newcommand{\Jc}{\mathcal{J}}
\newcommand{\E}{\vec{E}}
\newcommand{\B}{\vec{B}}
\newcommand{\Dp}{\frac{d^3p}{(2\pi)^3}}
\newcommand{\Dk}{\frac{d^3k}{(2\pi)^3}}
\newcommand{\Lv}{\mathbf{L}}
\newcommand{\Kv}{\mathbf{K}}
\newcommand{\Lambdahalf}{\Lambda_{\frac{1}{2}}}
\newcommand{\laout}{_\text{out}\la}
\newcommand{\rain}{\ra_\text{in}}
\newcommand{\bv}{\mathbf{b}}
\newcommand{\ddk}{\dfrac{d^d k}{(2\pi)^d}}
\newcommand{\dk}[1]{\dfrac{d^{#1} k}{(2\pi)^{#1}}}
\newcommand{\ddl}{\dfrac{d^d l}{(2\pi)^d}}
\newcommand{\dl}[1]{\dfrac{d^{#1} l}{(2\pi)^{#1}}}
\newcommand{\kslash}{\slashed k}
\newcommand{\nslash}{\slashed n}
\newcommand{\pslash}{\slashed p}
\newcommand{\qslash}{\slashed q}
\newcommand{\lslash}{\slashed l}
\newcommand{\vslash}{\slashed v}
\newcommand{\Dslash}{\slashed D}
\newcommand{\Aslash}{\slashed A}
\newcommand{\eslash}{\slashed \epsilon}
\newcommand{\cf}[2]{c_{#1}^{(#2)}}
\newcommand{\gs}{\text{g.s.}}
\newcommand{\g}[1]{\gamma^{#1}}
\newcommand{\vac}{\text{vac}}
\newcommand{\thetabar}{\bar{\theta}}
\newcommand{\etabar}{\bar{\eta}}
\newcommand{\hov}{\bar{h}}
\newcommand{\Pv}{P_\vslash}
\newcommand{\phicl}{\phi_\text{cl}}
\newcommand{\Veff}{V_\text{eff}}
\newcommand{\C}{\mathbb{C}}
\DeclareMathOperator{\Tr}{Tr}
\DeclareMathOperator{\Trd}{tr} %dirac trace
\DeclareMathOperator{\Li}{Li}
\DeclareMathOperator{\U}{U}
\DeclareMathOperator{\SU}{SU}
\DeclareMathOperator{\hc}{h.c.}
\DeclareMathOperator{\erf}{erf}
\DeclareMathOperator{\csch}{csch}
\DeclareMathOperator{\SO}{SO}
\newcommand{\LIPS}{\mathrm{LIPS}}
\newcommand{\F}{\mathcal{F}}
\newcommand{\bigv}{\vphantom{\begin{pmatrix}a\\b\end{pmatrix}^T}}
\newcommand{\fd}[1]{\left[d#1\right]}
\newcommand{\J}{\mathcal{J}}
\newcommand{\Y}{\mathcal{Y}}
\newcommand{\dop}[1]{\Delta_{#1}}
\newcommand{\Z}{\mathbb{Z}}
\newcommand{\Op}{\mathcal{O}}
\newcommand{\chibar}{\bar{\chi}}
\newcommand{\Ham}{\mathcal{H}}
\newcommand{\bipartial}{\barerset\leftrightarrow{\partial}}
\newcommand{\diff}[1]{\dfrac{d^3#1}{(2\pi)^{3/2}(2\omega_{#1})^{1/2}}}
\newcommand{\difft}[2]{\dfrac{d^3#1~d^3#2}{(2\pi)^{3}(2\omega_{#1})^{1/2}(2\omega_{#2})^{1/2}}}
\newcommand{\R}{\mathbb{R}}
\newcommand{\df}[2]{\dfrac{\delta #1}{\delta #2}}
\newcommand{\T}{\mathcal{T}}
\newcommand{\magpi}{|\vec p_i|}
\newcommand{\magpf}{|\vec p_f|}
\newcommand{\gev}{\text{GeV}}
\newcommand{\ev}{\text{eV}}
\newcommand{\nm}{\text{nm}}
\newcommand{\ns}{\text{ns}}
\newcommand{\mum}{\mu\text{m}}
\newcommand{\mm}{\text{mm}}
\newcommand{\cm}{\text{cm}}
\newcommand{\m}{\text{m}}
\newcommand{\km}{\text{km}}
\newcommand{\kev}{\text{keV}}
\newcommand{\mev}{\text{MeV}}
\newcommand{\tev}{\text{TeV}}
\newcommand{\ba}{\text{b}}
\newcommand{\nb}{\text{nb}}
\newcommand{\pb}{\text{pb}}
\newcommand{\gevp}{\text{GeV}/c}
\newcommand{\gevm}{\text{GeV}/c^2}
\newcommand{\msw}{\text{MSW}}
\newcommand{\cnub}{\text{C$\nu$B}}
\newcommand{\sun}{\text{Sun}}
\newcommand{\earth}{\text{Earth}}
\newcommand{\pow}[1]{\times 10^{#1}}
\newcommand{\theend}[1]{
    \begin{center}
    \vspace{#1mm}
    $\mathfrak{THE~END}$
    \end{center}
}
\newcommand{\el}{e^{-}}
\newcommand{\pos}{e^{+}}
\newcommand{\ord}[1]{\mathcal{O}(#1)}
\newcommand{\thus}{$~\Rightarrow~$}
\newcommand{\embedimg}[1]{\begin{center}\includegraphics[width=0.48\textwidth]{#1}\end{center}}
\newcommand{\embedimgw}[2]{\begin{center}\includegraphics[width=#2\textwidth]{#1}\end{center}}
\newcommand{\lrad}{L_\text{rad}}
\setcounter{section}{-1}
\nonstopmode
\begin{document}

\title{Notes for MIT NUPAX Oral Exam}
\date{Last modified: \today}
\author{Siddharth Narayanan, MIT}

\maketitle

\tableofcontents

\section{General information}

A barn is $10^{-24}~\cm^2$. Consider an incident flux $F$ on a point target (assuming the width of the beam is larger than the target). The differential cross-section is:
\begin{equation}
  \dd{\sigma}{\Omega}(E,\Omega) = \frac{1}{F} \dd{N_\text{scatt.}}{\Omega} \left[\frac{1}{\text{N/area}} \times \frac{\text{N}}{\text{solid angle}}\right]
\end{equation}
Given a flux $F$ incident on a target of area $A$, numberdensity $n$, and thickness $\delta x$, the number scattered is:
\begin{equation}
  N_\text{scatt.} = n\cdot A\cdot F\cdot \delta x \cdot \dd\sigma\Omega
\end{equation}
The probability-per-unit length for an interaction in matter is:
\begin{equation}
  \mu = \sigma\cdot N_A\rho/A
\end{equation}
where $A$ is the atomic mass
\noindent The natural lifetime is $\tau = \tau_{1/2}/\ln 2$.

\section{Basic radiation sources}
\begin{itemize}
  \item $\alpha$-decay: $(Z,A)\rightarrow(Z-2,A-4)+\alpha$
  \begin{itemize}
    \item Can be thought of as tunneling through nuclear potential
    \item $\Rightarrow$ narrow energy range ($\sim4-6~\mev$)
    \item Higher energy $\Rightarrow$ higher tunneling probability. Therefore, most $\alpha$-decays are to the ground state (highest $\Delta E$)
    \item Large charge: $\alpha$ particles only pass through few cm of air
  \end{itemize}
  \item $\beta$-decay: $n\rightarrow p+e^- + \nubar$, $p\rightarrow n+e^+ + \nu$
  \begin{itemize}
    \item Because $3$-body decay, there is a continuous spectrum for $e$ energy
    \item Typically goes to an excited nuclear state which decays by $\gamma$-radiation
    \item Electron capture: $p+\el\rightarrow n+\nu$
    \begin{itemize}
      \item Observed by looking for $\gamma$ from $\el$ filling hole left by captured $\el$
    \end{itemize}
  \end{itemize}
  \item $\gamma$-emission: nucleus de-excites, emitting a $\gamma$
  \begin{itemize}
    \item Selection rules: needs to be $\Delta S = 1$
    \item $\ord{0.1-1}~\mev$
    \item Typically happens quickly, but sometimes the transition is spin-suppressed (i.e. if need two $\gamma$s to get desired $\Delta S=2$ and $\Delta S=1$ state is higher energy)
    \item Spin-suppressed $\Rightarrow$ lifetimes up to $\ord 1 $ years
  \end{itemize}
  \item Annihilation: $\el+\pos\rightarrow2\gamma$, ($511~\kev$ each)
  \begin{itemize}
    \item $\pos$ from a $\beta^+$ process, which interacts with $\el$ in absorber or detector
    \item Signature is sharp peak at $E_\gamma = 511~\kev$. Photons are emitted back-to-back typically
  \end{itemize}
  \item Internal conversion: like $\gamma$-emission, but energy from nuclear de-excitation goes to an atomic $\el$ instead
  \begin{itemize}
    \item $\el$ is ejected from shell
    \item Typically $K$-shell electron
    \item Useful source of monoenergetic electrons for calibration
  \end{itemize}
  \item Neutron sources: $(Z,A)\rightarrow (Z_1,A_1)  + (Z_2,A_2) + n + n + \cdots$
  \begin{itemize}
    \item Fission products will undergo $\beta,\gamma$ decay themselves
    \item 
    \begin{equation}
      \dd{N}{E} = \sqrt{E} \exp \left[-\frac{E}{T}\right]
    \end{equation}
    where $E =$ energy of neutron, $T =$ characteristic energy for that decay $\sim\mev$
    \item Above is for \emph{spontaneous} fission (i.e. no extra energy added to nucleus to cause fission)
  \end{itemize}
  \item Nuclear reactions: $A+\alpha\rightarrow B+n$ or $A+\gamma \rightarrow B+n$
  \begin{itemize}
    \item e.g. $\alpha + ^9\text{Be} \rightarrow ^{13}\text{C}^*$, where $C^*$ decays by $n,\alpha$ or $3\alpha+n$
    \item Typically get $\ord{100}$ neutrons per $10^6$ $\alpha$s, depending on the source
    \item Neutron energy gets smeared quite a bit in decay
  \end{itemize}
\begin{center}
  \includegraphics[width=0.45\textwidth,valign=t]{figs/q_beta.png}
  \includegraphics[width=0.45\textwidth,valign=t]{figs/q_neutron.png}
\end{center}
\end{itemize}

\section{Passage of radiation through matter}
\begin{itemize}
  \item The probability of at least one interaction over a distance $x$ is $P(x) = 1-\exp[-n\sigma x]$, where $n$ is the number density of targets
  \begin{itemize}
    \item $\Rightarrow$ mean free path $\lambda = 1/N\sigma$
  \end{itemize}
\end{itemize}
\subsection{Energy loss of heavy charged particles in matter}
\begin{itemize}
  \item Anything heavier than $m_e$
  \item Interactions are with atomic $\el$ primarily: $\sigma \sim 10^7\text{-}10^8~\ba$
  \item Either excite (soft) or eject (hard) the $\el$
  \begin{itemize}
    \item $\el$ from hard interactions can cause secondary ionization
  \end{itemize}
  \item Quantity of interest is $-\la dE/dx\ra$, assuming large fluctuations are very unlikely (untrue for $e$)
  \item Bohr's classical calculation:
  \begin{itemize}
    \item Assuming the impact parameter between the charged particle (charge $z$) moving at velocity $v$ and the $\el$ is $b$:
    \begin{align*}
      I = \frac{2ze^2}{bv} &\Rightarrow \Delta E(b) = \frac{I^2}{2m_e} = \frac{2z^2e^4}{m_ev^2b^2}\\
      \Rightarrow - \dd{E}{x} & = \frac{4\pi z^2 e^4}{m_ev^2}N_e \ln \frac{\gamma^2 mv^3}{ze^2\nubar} \numberthis
    \end{align*}
    \item where $\nubar$ is the average frequency of bound state electrons. 
    \item The value of $dE/dx$ is gotten by integrating over a reasonable range for $b$ 
    \item Valid for very heavy particles (like $\alpha$), but not for protons, etc, because quantum effects
  \end{itemize}
  \item Bethe-Bloch:
  \begin{itemize}
    \item 
    \begin{equation}
      - \left\la \frac{dE}{dx} \right\ra \propto \rho \frac{Z}{A}\frac{z^2}{\beta^2} \left[\ln \left( \frac{2 m_e \gamma^2 \beta^2 W_{\max}}{I^2}\right) - 2\beta^2 - \delta - 2 \frac{C}{Z} \right]
    \end{equation}
    where:
    \begin{itemize}
      \item $W_\text{max}$ is the maximum energy transfer kinematically allowed
      \item $I$ is the mean excitation potential (material-dependent)
      \item $\delta$ is the density correction (at high $\beta$). Corrects for the charged particle polarizing the medium as it travels through (stronger effect for high density). This cancels the quadratic rise from the $\beta^2$ term
      \item $C/Z$ is the shell-effect. Corrects for case when the incident particle is slow relative to the electron orbital velocity
    \end{itemize}
    \item Minimum ionizing particle occurs at $\beta \sim 0.96 \Rightarrow \gamma \sim 3.6$ (independent of the particle)
    \begin{center}
      \includegraphics[width=0.8\textwidth]{figs/dEdx_bbPDG.png}
      \includegraphics[width=0.8\textwidth]{figs/dEdx_bethebloch.png}
    \end{center}
    \item Note that the particle is more ionizing at low $E$ \thus more ionization occurs at end of path
    \embedimgw{figs/dEdx_bragg.png}{0.8}
    \item For a fixed medium:
    \begin{itemize}
      \item $\dd{E}{x} = z^2 f(\beta)$
      \item \thus if $\dd{E}{x}$ is known for one particle, it is known for all particles (just have to scale $z^2$ and calculate new $\beta$ as a function of energy, mass)
    \end{itemize}
  \end{itemize}
\end{itemize}

\subsection{Cerenkov radiation}
\begin{itemize}
  \item Speed of light in medium is $c/n$
  \item If $\beta>1/n$, a cone of light is observed at an angle $\theta_C$
  \item $\cos\theta_C = (\beta n(\nu))\inv$ ($\nu$ is the frequency of the radiation)
  \item The energy loss is:
  \begin{gather*}
    - \dd{E}{x} = z^2 \frac{\alpha \hbar}{c} \int d\nu \left[\nu \left(1-(\beta n(\nu))^{-2}\right) \right]\\
    \Longleftrightarrow \\
    \frac{d^2N_\gamma}{d\nu dx} = \frac{z^2\alpha}{c}\left(1-(\beta n(\nu))^{-2}\right) \numberthis
  \end{gather*}
\end{itemize}

\subsection{Interactions of electrons/positrons in matter}
\begin{itemize}
  \item Bremsstrahlung becomes dominant at $E>E_c~\sim\ord{10}~\mev$. Below this, collisions are dominant (described by Bethe-Bloch)
  \item Bremsstrahlung cross-section is $\propto (e^2/m)^2$ \thus for particles with high $m$, bremsstrahlung doesn't become relevant until very very high $E$
  \begin{itemize}
    \item Kicks in at lower $E$ for $e^\pm$
  \end{itemize}
  \embedimgw{figs/dEdx_electrons.png}{0.8}
  \item $E_c$ is defined as the energy at which $dE/dx$ due to collision and brem are the same. An empirical formula is:
  \begin{equation}
    E_c = \frac{800~\mev}{Z+1.2}
  \end{equation}
  \item Radiation length: distance over a particle's energy is reduced by $1/e$. Call it $\lrad$
  \begin{itemize}
    \item $E=E_0 \exp[-x/\lrad]$
  \end{itemize}
  \begin{center}
    \includegraphics[width=0.45\textwidth]{figs/ecrit_table.png}
    \includegraphics[width=0.45\textwidth]{figs/radlength_table.png}
  \end{center}
  \item Because of low mass, variance in $dE/dx$ is larger for $e^\pm$ than for heavier particles (long tail from large energy transfers)
\end{itemize}

\subsubsection{Multiple Coulomb scattering}
\begin{itemize}
  \item Elastic scattering off nuclei
  \item Very little energy loss since $m_{A} \gg m$ typically
  \item Rutherford formula:
  \begin{equation}
    \dd{\sigma}{\Omega} = z^2Z^2 \left(\frac{\alpha\hbar c}{T}\right)^2 \frac{1}{\sin^4(\theta/2)}
  \end{equation}
  \begin{itemize}
    \item $T$ is the kinetic energy
    \item Largest probability for small deflections
  \end{itemize}
  \item Multiple ($N_\text{scatt.}>20$) scatterings can be treated statistically
  \begin{itemize}
    \item Assume $P(\theta)$ is Gaussian centered at $0$ (this ignores large deflections in the tails)
    \item $\la \theta^2\ra \sim 10^{-3}~\text{rad}^2$ for typical materials
    \item Not valid for $e^{\pm}$ which have low mass \thus large deflections (like backscattering) possible
  \end{itemize}
\end{itemize}

\subsection{Interaction of photons in matter}

\subsubsection{Photoelectric effect}
\begin{itemize}
  \item $\gamma$+atom $\rightarrow$ ion+$\el$
  \item $E_e$ = $h\nu - \phi$, $\phi$ is the binding energy
  \item $\sigma$ decreases as $h\nu$ increases, with a jump up when $h\nu$ reaches a shell binding energy
  \item $\sigma\sim \ord{10^{-1}-10^{4}}~\ba$ and is proportional to $Z^\alpha$, for $\alpha=4-5$ (exact value is a function of $h\nu$)
\end{itemize}
  \embedimgw{figs/sigma_pe.png}{.8}

\subsubsection{Compton scattering}
\embedimgw{figs/compton_feynman.png}{0.4}
\begin{itemize}
  \item Kinematics
    \begin{gather*}
      \lambda'-\lambda = \frac{1}{\nu'} - \frac{1}{\nu} = \frac{h}{m_e} \left(1-\cos\theta\right)\\
      \cot\phi = \left(1+\frac{h\nu}{m_e}\right)\tan \frac{\theta}{2}
    \end{gather*}
  \item Klein-Nishina:
    \begin{align*}
      \dd\sigma\Omega &= \frac{\alpha^2}{2m_e^2} \left(\frac{\nu'}{\nu}\right)^2 \left[\frac{\nu'}{\nu}+\frac{\nu}{\nu'} - \sin^2\theta\right] \numberthis \\
       \sigma & \sim 10^{-4} - 10^0~\ba
    \end{align*}
  \item Can define $\sigma_S$ ($\sigma_A$) as the cross-section of scattered (absorbed) energy:
    \begin{equation}
      \dd{\sigma_S}{\Omega} = \frac{h\nu'}{h\nu}\dd\sigma\Omega, \quad \sigma_A = \sigma-\sigma_S
    \end{equation}
    \embedimgw{figs/sigma_compton.png}{.8}
  \item Can also define $d\sigma/dT$, where $T$ is the kinetic energy of $\el$:

  \begin{minipage}{.4\textwidth}
    \embedimgw{figs/dEdT_compton.png}{.8}
  \end{minipage}
  \begin{minipage}{.6\textwidth}
      $0 < T < T_\text{max}$, where $T_{\max}$ is the max allowed recoil energy:
      \begin{equation}
        T_{\max} = h\nu \frac{2\xi}{1+2\xi},~\xi = \frac{h\nu}{m_e}
      \end{equation}
  \end{minipage}
\end{itemize}

\subsubsection{Pair production}
\begin{itemize}
  \item $\gamma\rightarrow \el\pos$
  \item Mean free path: $\lambda_\text{pair} = \frac{9}{7}L_\text{rad}$, where $L_\text{rad}\sim \ord{10}~\cm$ is for electrons
  \embedimgw{figs/sigma_pair.png}{.4}
\end{itemize}

\subsubsection{Photon interaction summary}

\embedimgw{figs/sigma_photonInclusive.png}{.9}

\subsubsection{EM showers}
\begin{itemize}
  \item $\gamma\rightarrow\el\pos$, $e^\pm\rightarrow e^\pm+\gamma$, $\gamma\rightarrow\el\pos\cdots$
  \item Assuming the radiation lengths are $L_\gamma\sim L_e$:
  \begin{itemize}
    \item $N_\text{particles}\sim 2^{x/L}$
    \item $E(x)\sim E_0/2^{x/L}$, where $E$ is the energy per particle in the shower at distance $x$
    \item Obviously has to truncate when $E(x)\leq m_e$, but actually stops at $E<E_c$ (no more brem)
    \item Let $t=x/L$. Then, $t_{\max} = \ln(E_0/E_c)/\ln2$ and $N_{\max}\approx E_0/E_c$
  \end{itemize}
\end{itemize}

\subsection{Interactions of neutrons in matter}
\begin{itemize}
  \item High energy neutrons: $E>100~\mev$
  \item Fast: $\ord{100}~kev < E < \ord{10}~\mev$
  \item Epithermal: $\ord{0.1}~\ev < E <\ord{100}~\kev$ (where nuclear resonances occur)
  \item Thermal: $E\sim 1/kT \approx 1/40~\ev$
  \item Ultracold: $E<0.001~\ev$
\end{itemize}
\subsubsection{Mechanisms of interaction}
\begin{itemize}
  \item $A+n\rightarrow A+n$: dominant for $\mev$ neutrons (elastic)
  \item $A+n\rightarrow A^*+n'$, $A+n\rightarrow B+2n',\dots$
  \begin{itemize}
    \item Neutron must be $> 1~\mev$ to excite the nucleus
    \item Inelastic
  \end{itemize}
  \item Radiative $n$ capture: $n+(Z,A)\rightarrow \gamma+(Z,A+1)$
  \begin{itemize}
    \item $\sigma \sim 1/v$, so dominant at low energies
    \item There can be resonance peaks
  \end{itemize}
  \item $A+n\rightarrow B + p/d/\alpha/t/\alpha p/\cdots$
  \begin{itemize}
    \item Typically $\ev-\kev$
    \item $\sigma\sim 1/v$
  \end{itemize}
  \item Fission (more likely at low [thermal] energies)
  \item High energy hadron shower
  \begin{itemize}
    \item Similar to EM shower
    \item $E > 100~\mev$
  \end{itemize}
\end{itemize}
\subsubsection{Neutron moderation}
\begin{itemize}
  \item Neutrons bounce around matter, slowing down until at thermal eq. w/ matter
  \begin{itemize}
    \item Unlikely to be captured by nucleus (or cause fission) at high energies because $\sigma\sim 1/v$ for these processes
  \end{itemize}
  \item Energy of scattered neutron:
  \begin{equation}
    \left(\frac{A-1}{A+1}\right)^2 E_0 < E' <E_0
  \end{equation}
  which implies low $A$ is better for absorbing energy
\end{itemize}

\section{General characteristics of detectors}
\begin{itemize}
  \item Sensitivity to signal is a function of many things:
  \begin{itemize}
    \item $\sigma$ of ionizing reactions in the detector
    \item Detector mass
    \item Material surrounding sensitive volume of detector
  \end{itemize}
  \item Two continuous responses are considered resolved if separated by a distance greater than the FWHM
  \item Error is Poissonian for detectors that collect \emph{some} of the particle's energy
  \begin{itemize}
    \item Error is smaller for detectors that \emph{stop} the particle, since each interaction is not independent
  \end{itemize}
  \item Define $J=E/W$, so the energy variance is $\sigma^2=FJ$
  \begin{itemize}
    \item $W=$energy lost per ionization in detector
    \item $F=$Fano factor
    \begin{itemize}
      \item $F=1$ \thus purely Poissonian (e.g. scintillators in which only part of the energy is deposited)
      \item $F<1$ \thus better (semiconductors, gases, scintillators which stop the particle)
    \end{itemize}
  \end{itemize}
\end{itemize}

\section{Ionization detectors}
\subsection{Types of gaseous ionization detectors}
\embedimgw{figs/gasionization.png}{.8}
\begin{itemize}
  \item Filled with a noble gas
  \item Radial $\E$-field from anode wire to cathode ($\propto 1/\ln(b/a)$, where $a$ is the wire radius and $b$ is the chamber radius)
  \item Radiation creates $\el$/ion pairs in tube
  \begin{itemize}
    \item If the radiation is charged, this occurs through ionization
    \item If neutral, it occurs through secondary ionization (i.e., $\gamma,n$ ejects an $\el$ from an atom, which then ionizes the gas)
    \item Number of pairs is $\propto E$
    \item $\el\rightarrow$anode, ions$\rightarrow$cathode
    \item Observed signal is a function of $V_0$:
    \embedimgw{figs/ionization_voltage}{0.8}
  \end{itemize}
\end{itemize}
\subsubsection{Ionization chamber}
\begin{itemize}
  \item Number of $\el$/ion pairs produced is equal to number of ionizations by particle
  \item Current is typically quite low
  \item Can be used for experiments with large radiation fluxes
\end{itemize}
\subsubsection{Proportional counter}
\begin{itemize}
  \item $\el$ from primary ionization drifts towards anode ($v\sim \mum/\ns$)
  \item $\E$-field is strong enough close to the anode that $\el$s from ionization are accelerated and cause secondary ionizations
  \item These secondary ionization $\el$s can also create tertiary ionizations, and so forth
  \item Occurs in the ``avalanche region'' close to the anode (since $|\vec E(r)|\propto 1/\ln(r/a)$), typically $r-a \lesssim 10~\mum$
  \item Up to $10^6$ amplification
\end{itemize}
\subsubsection{Geiger-M\"uller counter}
\begin{itemize}
  \item $\Delta V$ is so high that an ionization sparks multiple avalanches along entire length of anode
  \begin{itemize}
    \item $\el$s from primary ionization are accelerated so strongly that when they hit an atom, they ionize it but also put the atom in an excited state
    \item When the atom de-excites, it releases a $\gamma$, which initiates its own avalanche
    \item Output current is independent of $E$
    \item Need a quenching gas to absorb $\gamma$s and shorten the signal pulse
  \end{itemize}
  \item Obviously only useful for counting number of incident particles
  \item If $V_0 > 1~\text{kV}$, you get spontaneous breakdown of the gas, and it is no longer useful
\end{itemize}

\subsection{Ionization and transport in gas}
\begin{itemize}
  \item Ionization mechanisms (assuming $\psi$ is some charged particle):
  \begin{itemize}
    \item $\psi+X\rightarrow\psi+X^*$ is a resonant process
    \begin{itemize}
      \item In noble gases, $\sigma\sim 10^7~\ba$
    \end{itemize}
    \item $\psi+X \rightarrow \psi+X^++\el$ is not resonant
    \begin{itemize}
      \item $\sigma\sim 10^8~\ba$
      \item Higher energy threshold than excitations (need to overcome valence binding energy)
    \end{itemize}
  \end{itemize}
  \item In general, $\sigma$ is higher for low-energy transfers, so excitations dominate ionizations
  \item $\el$/ion pairs produced by $\psi$ can also create secondary $\el$/ion pairs if energetic enough
  \begin{itemize}
    \item Process repeats until $E_{\el}<$ionization threshold
  \end{itemize}
\end{itemize}
\subsubsection{Mean number of $\el$/ion pairs}
\begin{itemize}
  \item Note that the number of pairs is not equal to (energy lost)/(ionization energy), because some energy is lost through excitation of atoms
  \item In noble gases, 1 $\el$/ion pair produced corresponds to $\sim30~\ev$ energy lost
  \item For gaseous detectors, the Fano factor is typically $\sim 0.2$
\end{itemize}
\subsubsection{Recombination and $\el$-attachment}
\begin{itemize}
  \item These processes eat up $\el$/ion pairs before detection
  \item Recombination: if $\psi+X\rightarrow \psi+X^++\el$, then we can have $X^++\el \rightarrow X+\gamma$
  \begin{itemize}
    \item Similarly, if two ions exist in the gas: $X^++Y^- \rightarrow XY+\gamma$
  \end{itemize}
  \item Electron attachment: capture by electronegative atoms: $\el+Y \rightarrow Y^- + \gamma$
\end{itemize}
\subsubsection{Diffusion}
\begin{itemize}
  \item In principle, a charged particle in a detector will follow the $\E$-field lines
  \item But, it can be knocked about through elastic collisions with atoms
  \item Problem for detectors in which electrons must drift very far
  \item One way to minimize the effect is to add a parallel $\B$-field (so that if a particle is knocked off the field line, it remains in a tight helix). This is done in TPCs
\end{itemize}

\subsection{Avalanche multiplication}
\embedimg{figs/avalanche.png}
\begin{itemize}
  \item The initial $\el$ has high energy and causes secondary, tertiary, etc ionizations
  \item $\el$s move much faster than ions, giving the charge distribution a liquid drop shape
  \item Let $\lambda$ be the mean free path of an $\el$ causing secondary ionization
  \begin{itemize}
    \item $\alpha = \lambda\inv$ is the prob. of ionization/unit length
    \begin{equation}
      dn = n\alpha dx
    \end{equation}
    where $n$ is the number of free $\el$. Then:
    \begin{equation}
      n = n_0 e^{\alpha x}
    \end{equation}
    where $n_0$ is the number of primary ionizations
    \item $e^{\alpha x}$ is the proportionality factor in prop. counters
  \end{itemize}
  \item Above calculation assumes a constant $\E$-field \thus constant $\alpha$
  \begin{itemize}
    \item More generally:
    \begin{equation}
      e^{\alpha x}\rightarrow \exp \left[\int_{r_1}^{r_2} dx~\alpha(x)\right]
    \end{equation}
  \end{itemize}
\end{itemize}

\subsection{Pulse formation and shape}
\begin{itemize}
  \item Consider a cylindrical proportional counter as an example
  \item Consider the counter as a coax capacitor with capacitance/unit length $C$
  \begin{itemize}
    \item If a charge $q$ moves by $dr$, then 
    \begin{equation}
      dW = q\dd{\phi}{r} dr
    \end{equation}
    where $W$ is the system energy and $\phi$ is the electric potential
    \item For a capacitor of length $L$, $W = \frac{1}{2}LCV_0^2$, so
    \begin{equation}
      dW = LCV_0 dV
    \end{equation}
    \item Equating:
    \begin{equation}
      dV = \frac{q}{LCV_0}\phi'(r) dr
    \end{equation}
    \item That is, the motion of $q$ by $dr$ induces a measurable change in the potential 
  \end{itemize}
  \item \thus observed signal is \emph{not} charges collecting on the wire/wall, but an induced voltage change due to moving charges
  \item The total induced voltage change for a particle starting a distance $r'$ from the wire is:
  \begin{equation}
    \int dV = \int_{a+r'}^{R} dr~\frac{q}{LCV_0}\phi'(r)
  \end{equation}
  where $R = a$ ($R=b$) for $\el$ (ions). 
  \begin{itemize}
    \item Avalanches only occur when the $\E$-field is very high (close to the anode), so typically $r\sim\ord(1)~\mum$ ($a\sim\mum,b\sim\mm$)
    \item i.e. $b-(a+r')\gg r'$ \thus pulse is primarily caused by motion of positive charges
    \item For many (but not all!) types of prop. counters, can ignore signal from $\el$
  \end{itemize}
  \item In the following figure, time starts with the beginning of the avalanche (drift time is ignored)
  \embedimgw{figs/propcounter_pulseshape.png}{.8}
\end{itemize}

\subsection{Multiwire proportional chamber}


\begin{appendices}
\section{Bibliography}

Leo, W. R., \emph{Techniques for Nuclear and Particle Physics Experiments}. Second Revised Edition
\vspace{5mm}

\noindent PDG Particle Review, \emph{Passage of Particles Through Matter}. 2014 edition

\end{appendices}

\end{document}
